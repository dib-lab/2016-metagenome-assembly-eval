%%%%%%%%%%%%%%%%%%%%%%%%%%%%%%%%%%%%%%%%%%%%%%%%%%%%%%%%%%%%%%%
%
% Welcome to Overleaf --- just edit your article on the left,
% and we'll compile it for you on the right. If you give 
% someone the link to this page, they can edit at the same
% time. See the help menu above for more info. Enjoy!
%
%%%%%%%%%%%%%%%%%%%%%%%%%%%%%%%%%%%%%%%%%%%%%%%%%%%%%%%%%%%%%%%
%
% For more detailed article preparation guidelines, please see:
% http://f1000research.com/author-guidelines

\documentclass[10pt,a4paper,twocolumn]{article}
\usepackage{f1000_styles}
\usepackage{multirow} 
%% Default: numerical citations
\usepackage[numbers]{natbib}

%% Uncomment this lines for superscript citations instead
% \usepackage[super]{natbib}

%% Uncomment these lines for author-year citations instead
% \usepackage[round]{natbib}
% \let\cite\citep

\begin{document}

\title{\textit{F1000Research} Evaluating Metagenome Assembly on a Complex Community}
\titlenote{ }
\author[1]{Sherine Awad}
\author[2]{Luiz Irber}
\author[3]{C. Titus Brown}
\affil[1,2,3]{Department of Population Health and Reproduction, University of California, Davis, California}
 

\maketitle
\thispagestyle{fancy}

% Please list all authors that played a significant role in the research involved in the article. Please provide full affiliation information (including full institutional address, ZIP code and e-mail address) for all authors, and identify who is/are the corresponding author(s).

\begin{abstract}
 

 
Metagenome assembly is a challenging problem due to the biodiversity
of the microorganisms. Most assemblers are designed for whole genome
assembly and not capable of dealing with metagenomic samples. However,
in order to decide which assembler works best for metagenome, we need
to evaluate metagenome assembly generated by each assembler.

% CTB: mention microdiversity, species foo.

In this paper, we used three assemblers ; IDBA-UD, SPAdes, and Megahit
to assemble metagenome mock community data and evaluate the assembly
process in terms of resources utilization, assembly quality, genome
fraction covered, duplication ratio, misassemblies and partial
alignments.

The results show only small differences in content recovery between
assemblers. However, Megahit is much faster and produces shorter
contig lengths than IDBA-UD and SPAdes.
 


%Motivation: With the emergence of de novo assembly, several work have been done to assemble metagenomic data from de novo. Several assemblers exist that are based on different assembly techniques. However, we still lack  a study that analyze different assemblers behaviors on metagenomic data. 
 
%Problem statement: In this paper, we performed an analytic study for metagenome assembly using different assemblers. The aim of the analysis is studying how well metagenome assembly works, and which assembly works best.  


%Approach: We used a mock community dataset for the analysis, and used its reference genome for the benchmark evaluation. We quality filtered the reads  then assembled the reads using three different assemblers: IDBA-UD, SPAdes, and MEGAHIT.


 

\end{abstract}

\clearpage

\section*{Introduction}

Metagenomics refers to sequencing of DNA from a mixture of organisms,
often from an environmental or uncultured sample. Unlike whole genome
sequencing, metagenomics targets a mixture of genomes, which
introduces metagenome-specific challenges in analysis.  Most
approaches to analyzing metagenomic data rely on mapping or comparing
sequencing reads to reference sequence collections. However, reference
databases contain only a small subset of microbial diversity (cite:
geba), and the much of the remaining diversity is evolutionarily distant
and search techniques may not access it.

As sequencing capacity increases and sequence data is generated from
many more environmental samples, metagenomics is increasingly using de
novo assembly techniques to generate new reference genomes and
metagenomes.  There are a number of metagenome assemblers that are
widely used. However, evaluating the results of these assemblers is
challenging due to the general lack of good quality reference
metagenomes.  Below, we evaluate three commonly assemblers - SPAdes,
IDBA, and MEGAHIT - on a mock community containing 64 species of
microbes with known genomes.

% @CTB reference megahit 1.0 paper which did an assembly and evaluation
% of the same data set as us.

 
 %-----------Literature Review starts here ---------------------
 
Moya et al. in \cite{moya2014} evaluated metagenome assembly using
simulated two 454 viral metagenome and six assemblers. The assemblies
were evaluated based on several metrics including N50, percentages of
reads assembled, accuracy when compared to the reference genome. In
addition to, chimeras per contigs and the effect of assembly on
taxonomic and functional annotations.
 
Mavromatis et al. in \cite{mavromatis2007} provided a benchmark study
to evaluate the fidelity of metagenome process methods. The study used
simulated metagenomic data sets constructed at different complexity
levels.
%The low complexity set is dominated by a single near-clonal organism. The medium complexity set includes moderate communities with more than one dominant organism. The high complexity set lacks dominant population.  
The datasets were assembled using Phrap v3.57, `Arachne v.2
\cite{arachne} and JAZZ. \cite{jazz}

The study evaluates assembly, gene prediction, and binning
methods. However, the study did not evaluate the assembly quality
against a reference genome.

Rangwala et al. in \cite{huzefa2011} presented an evaluation study of
metagenome assembly. The study used a de Bruijn graph based assembler
ABYSS \cite{abyss} to assemble simulated metagnome reads of 36 bp. The
data set is classified at different complexity levels.
%The data set is classified into 3 classes, low complexity data set in which the reads belongs to a single dominant organism, a medium complexity data set, in which the reads has more than one dominant organism with lower concentration, and the high complexity data set which has no distinct dominant organism.
The study compares the quality of the assembly of the data sets in
terms of quality measures of contigs length, assembly accuracy. The
study also took into consideration the effect of kmer size and the
degree of chimericity.  However, the study evaluated the assembly
based on one assembler, and did not evaluate assembly against several
assemblers.  Also, both previous studies used simulated data, which
may lack confounders of assembly such as sequencing artifacts and GC bias.
 
Lindgreeb et al in \cite{metaclass} presented a benchmark study for
metagenome analysis tools. The authors compared between several
metagenome classification tools in terms of run time, ease of use,
information provided, reads and shuffled reads mapped, non existing
phyla, divergence of real distribution, and correlation with known
community composition. However, the paper did not consider metagenome
assembly tools in the study.

% @CTB mention shakya paper in metagome analysis.
 
In this paper, we evaluate metagenome assembly on the data set from
Shakya et al. (2013) using three assemblers; IDBA-UD \cite{idba},
SPAdes \cite {spades}, and MEGAHIT \cite{megahit}.

SPAdes \cite{spades} is an assembler for both single-cell and standard
(multicell) assembly. (More description here @CTB.)

IDBA-UD \cite{idba} is a de Bruijn graph
approach for assembling reads from single cell sequencing or
metagenomic sequencing technologies with uneven sequencing
depths. IDBA-UD uses multiple depth-relative thresholds to remove
erroneous k-mers in both low-depth and high-depth regions. It also
uses paired-end information to solve the branch problem of low-depth
short repeat regions. It also applies an error correction step to correct
reads of high-depth regions that can be aligned to high confident
contigs.

MEGAHIT \cite{megahit} is a newer approach that constructs a succinct
de Bruijn graph using multiple k-mer sizes, and uses a novel ``mercy
k-mer'' approach that preserves low-abundance regions of reads. It also
can use GPUs to accelerate the graph construction.

We evaluate the performance of the three assemblers using real mock
community data. The performance of each assembler is compared in terms
of resources utilization, covered genome fraction, duplication ratio,
misassemblies, and contig length. This helps decide which assembler to
use when we lack a reference.

% \subsection*{Sections}

% Use section and subsection commands to organize your document. \LaTeX{} handles all the formatting and numbering automatically. Use ref and label commands for cross-references.


% Removed this line for now \subsection*{Tables}

% Use the table and tabledata commands for basic tables --- see Table~\ref{tab:widgets}, for example.
% \begin{table}[h!]
% \hrule \vspace{0.1cm}
% \caption{\label{tab:widgets}An example of a simple table with caption.}
% \centering
% \begin{tabledata}{llr} 
% \header First name & Last Name & Grade \\ 
% \row John & Doe & $7.5$ \\ 
% \row Richard & Miles & $2$ \\ 
% \end{tabledata}
% \end{table}
 
 %------------------------------------Cost of Assembly ---------------
 \begin{table}[h]
\caption{Running Time and Memory Utilization}
\centering
\begin{tabular}{|c|c|}
\hline
\multicolumn{2}{|c|}{ \textbf{(1) IDBA-UD}}    \\ [0.5ex] % inserts table %heading
\hline
\textbf{Running Time}&17:12:43 \\ 
\hline
\textbf{Memory Utilization (GB)}&149.12\\ 
\hline
\multicolumn{2}{|c|}{ \textbf{(2) SPAdes} }   \\ [0.5ex] % inserts table %heading
\hline
\textbf{Running Time} & 42:14:06   \\
\hline
\textbf{Memory Utilization (GB)}& 391.45   \\ 
\hline
\multicolumn{2}{|c|}{ \textbf{(3) MEGAHIT} }    \\ [0.5ex] % inserts table %heading
\hline
\textbf{Running Time}& 56:04.43 \\
\hline
\textbf{Memory Utilization (GB)}& 34.40 \\ 
\hline

\end{tabular}
\label{table:time-memory}
\end{table}

 %------------------------------Coverage Table-----------------------------
\begin{table}[h!]
\caption{Reference Genome Coverage and Duplication Ratio}
\centering
\begin{tabular}{|c|c|c|}
 \hline 
 \multicolumn{3}{|c|} {\textbf{(1) Best hit Approach}}    \\ [0.5ex] % inserts table 
 \hline
\multicolumn{3}{|c|}{ \textbf{(1) IDBA-UD}}    \\ [0.5ex] % inserts table %heading
\hline
\multirow{2}{*}{99.0}&\textbf{Genome  Coverage} & 56.89 \%     \\   
&\textbf{Duplication Ratio} & 0.38 \%   \\   
\hline
\multirow{2}{*}{95.0}&\textbf{Genome  Coverage}& 58.00\%     \\   
&\textbf{Duplication Ratio}& 0.59 \%    \\   
\hline
\multicolumn{3}{|c|}{ \textbf{(2) SPAdes} }   \\ [0.5ex] % inserts table %heading
\hline
\multirow{2}{*}{99.0}&\textbf{Genome Coverage}&  63.79 \%  \\
&\textbf{Duplication Ratio} & 0.15 \% \\   
\hline 
\multirow{2}{*}{95.0}&\textbf{Genome Coverage}&64.68 \%  \\
&\textbf{Duplication Ratio}& 0.26\%  \\   
\hline
\multicolumn{3}{|c|}{ \textbf{(3) MEGAHIT} }    \\ [0.5ex] % inserts table %heading
\hline
\multirow{2}{*}{99.0}&\textbf{Genome  Coverage}&68.47  \% \\
&\textbf{Duplication Ratio}&0.37 \% \\   
\hline
\multirow{2}{*}{95.0}&\textbf{Genome  Coverage}& 68.96 \%  \\
&\textbf{Duplication Ratio}& 0.45\%  \\   
\hline
 \multicolumn{3}{|c|} {\textbf{(2) Ambiguous Approach}}    \\ [0.5ex] % inserts table %heading
\hline
\multicolumn{3}{|c|}{ \textbf{(1) IDBA-UD}}    \\ [0.5ex] % inserts table %heading
\hline
\multirow{2}{*}{99.0}&\textbf{Genome Coverage} &89.79 \%   \\   
&\textbf{Duplication Ratio} &0.94\%  \\   
\hline
\multirow{2}{*}{95.0}&\textbf{Genome  Coverage} &95.46  \%  \\   
&\textbf{Duplication Ratio} &1.90 \%    \\   
\hline
\multicolumn{3}{|c|}{ \textbf{(2) SPAdes} }   \\ [0.5ex] % inserts table %heading
\hline
\multirow{2}{*}{99.0}&\textbf{Genome Coverage}&89.42 \% \\
&\textbf{Duplication Ratio}&1.00 \%  \\   
\hline
\multirow{2}{*}{95.0}&\textbf{Genome Coverage}&95.12 \%  \\
&\textbf{Duplication Ratio}&1.98 \%  \\   
\hline
\multicolumn{3}{|c|}{ \textbf{(3) MEGAHIT} }    \\ [0.5ex] % inserts table %heading
\hline
\multirow{2}{*}{99.0}&\textbf{Genome Coverage}&91.16 \%  \\
&\textbf{Duplication Ratio}&0.55 \%  \\   
\hline
\multirow{2}{*}{95.0}&\textbf{Genome Coverage} &94.22 \%  \\
&\textbf{Duplication Ratio}&1.48 \%   \\   
\hline
\multicolumn{3}{|c|} {\textbf{(3) No Misassemblies Approach}}    \\ [0.5ex] % inserts table %heading
\hline
\multicolumn{3}{|c|}{ \textbf{(1) IDBA-UD}}    \\ [0.5ex] % inserts table %heading
\hline
\multirow{2}{*}{99.0}&\textbf{Genome  Coverage} &34.60 \%   \\   
&\textbf{Duplication Ratio} & 0.37\%  \\   
\hline
\multirow{2}{*}{95.0}&\textbf{Genome  Coverage} & 35.22 \%  \\   
&\textbf{Duplication Ratio} & 0.56  \%    \\   
\hline
\multicolumn{3}{|c|}{ \textbf{(2) SPAdes} }   \\ [0.5ex] % inserts table %heading
\hline
\multirow{2}{*}{99.0}&\textbf{Genome Coverage}& 35.92\% \\
&\textbf{Duplication Ratio}&0.16 \%  \\   
\hline
\multirow{2}{*}{95.0}&\textbf{Genome Coverage}& 36.42 \%  \\
&\textbf{Duplication Ratio}& 0.21 \%  \\   
\hline
\multicolumn{3}{|c|}{ \textbf{(3) MEGAHIT} }    \\ [0.5ex] % inserts table %heading
\hline
\multirow{2}{*}{99.0}&\textbf{Genome Coverage}&45.81 \%  \\
&\textbf{Duplication Ratio}& 0.39\%  \\   
\hline
\multirow{2}{*}{95.0}&\textbf{Genome Coverage} & 46.03 \%  \\
&\textbf{Duplication Ratio}& 0.48 \%   \\   
\hline
\end{tabular}
\label{table:coverage-analysis}
\end{table}

 %------------------------------Contigs Length table-----------------------------

\begin{table}[h!]
\caption{Contigs Analysis}
\centering
\begin{tabular}{|c|c|c|}
\hline
 \multicolumn{2}{|c|} {\textbf{(1) Best hit Approach}}    \\ [0.5ex] % inserts table %heading
 \hline
\multicolumn{2}{|c|}{ \textbf{(1) IDBA-UD}}    \\ [0.5ex] % inserts table %heading
\hline
\textbf{No. of Contigs} & 19,988 \\   
\hline
\textbf{Totally Aligned Contigs \%} &72.96\%  (97,138,779)  \\   
\hline
\textbf{Partial Aligned Contigs \%} &10.97\%  (20,261,669)   \\   
\hline
\textbf{Unaligned Contigs \%} &  16.07\% (61,421,243)  \\   
\hline
\multicolumn{2}{|c|}{ \textbf{(2) SPAdes} }   \\ [0.5ex] % inserts table %heading
\hline
\textbf{No. of Contigs}&15,254 \\   
\hline
\textbf{Totally Aligned Contigs\%}&76.52\% (109,342,809)     \\   
\hline
\textbf{Partial Aligned Contigs\%}&12.08\% (22,008,234)  \\   
\hline
\textbf{Unaligned Contigs\%}&11.40\% (34,176,209)     \\   
\hline
\multicolumn{2}{|c|}{ \textbf{(3) MEGAHIT} }    \\ [0.5ex] % inserts table %heading
\hline
\textbf{No. of Contigs}&27,657\\   
\hline
\textbf{Totally Aligned Contigs \%}&83.07\%  (128,987,917)   \\   
\hline
\textbf{Partial Aligned Contigs\%}&4.67\%  (12,325,804) \\   
\hline
\textbf{Unaligned Contigs\%}&12.26\% (50,093,476) \\   
\hline
\multicolumn{2}{|c|} {\textbf{(2) Ambiguous  Approach}}    \\ [0.5ex] % inserts table %heading
\hline
\multicolumn{2}{|c|}{ \textbf{(1) IDBA-UD}}    \\ [0.5ex] % inserts table %heading
\hline
\textbf{No. of Contigs}& 19,988     \\   
\hline
\textbf{Totally Aligned Contigs\%}&80.59\% (161,075,933)   \\   
\hline
\textbf{Partial Aligned Contigs\%}& 8.54 \%   (22,638,415 )  \\   
\hline
\textbf{Unaligned Contigs \%}& 10.87\%  (13,378,572)    \\   
\hline
\multicolumn{2}{|c|}{ \textbf{(2) SPAdes} }   \\ [0.5ex] % inserts table %heading
\hline
\textbf{No. of Contigs} & 15,254  \\   
\hline
\textbf{Totally Aligned Contigs} & 81.03 \% (154,920,366)  \\   
\hline
\textbf{Partial Aligned Contigs\%}&9.34\%  (28,028,529) \\   
\hline
\textbf{Unaligned Contigs\%}&9.62\% (12,931,934)  \\   
\hline
\multicolumn{2}{|c|}{ \textbf{(3) MEGAHIT} }    \\ [0.5ex] % inserts table %heading
\hline
\textbf{No. of Contigs}&27,657\%  \\   
\hline
\textbf{Totally Aligned Contigs\%}&87.59 \% (169,789,173) \\   
\hline
\textbf{Partial Aligned Contigs\%}&4.36\% (15,658,616)  \\   
\hline
\textbf{Unaligned Contigs\%}&8.04\% (12,777,886)    \\   
\hline

\multicolumn{2}{|c|} {\textbf{(3) No Misassemblies  Approach}}    \\ [0.5ex] % inserts table %heading
\hline
\multicolumn{2}{|c|}{ \textbf{(1) IDBA-UD}}    \\ [0.5ex] % inserts table %heading
\hline
\textbf{No. of Contigs} &19,988 \\   
\hline
\textbf{Totally Aligned Contigs\%}&57.41\% (61,874,288)   \\   
\hline
\textbf{Partial Aligned Contigs\%}&6.81\%(9,530,232)  \\   
\hline
\textbf{Unaligned Contigs \%}&35.78\%  (122,899,406)    \\   
\hline
\multicolumn{2}{|c|}{ \textbf{(2) SPAdes} }   \\ [0.5ex] % inserts table %heading
\hline
\textbf{No. of Contigs} &15,254 \\   
\hline
\textbf{Totally Aligned Contigs} &62.93\% (64,656,150)  \\   
\hline
\textbf{Partial Aligned Contigs\%}&6.84\% (9,323,121) \\   
\hline
\textbf{Unaligned Contigs\%}&30.23\% (114,292,858)  \\   
\hline
\multicolumn{2}{|c|}{ \textbf{(3) MEGAHIT} }    \\ [0.5ex] % inserts table %heading
\hline
\textbf{No. of Contigs}&27,657 \\   
\hline
\textbf{Totally Aligned Contigs\%}&67.62\%  (87,488,973) \\   
\hline
\textbf{Partial Aligned Contigs\%}&2.73\% (7,069,258)  \\   
\hline
\textbf{Unaligned Contigs\%}&29.65\% (102,672,613)    \\   
\hline
\end{tabular}
\label{table:contigs-analysis}
\end{table}


%------------------------ List of Genomes of mostly uncovered bases-----------------
\begin{table}[!h]
 \centering
 \caption{Genomes with the most common uncovered bases between the three assemblies.}
 \begin{tabular}{|p{4.50cm}|p{2.75cm}|} \hline
 \textbf{Genome} & \textbf{Uncovered bases (\%~of~total)} \\ \hline 
Shewanella baltica OS223 & 2.3 Mbp (18.25\%) \\ \hline
Fusobacterium nucleatum & 2.2 Mbp (16.95\%) \\ \hline
Desulfovibrio vulgaris DP4 & 1.6 Mbp (12.85\%) \\ \hline
Enterococcus faecalis V583 & 1.0 Mbp (7.83\%) \\ \hline
Thermus thermophilus HB27 & 0.7 Mbp (5.73\%) \\ \hline
\end{tabular}
\label{table:genomes_uncovered}
\end{table}
%---------------------------------N50-NG50 table 

%\begin{table}[!ht]
%\caption{Comparison between N50 and NG50}
%\centering
%\begin{tabular}{|c|c|c|}
%\hline

%\textbf{Assembler}& \textbf{N50} &  \textbf{NG50}    \\ [0.5ex] % inserts table %heading
%\hline
%\textbf{IDBA-UD}&55,225 &52,031\\
%\hline
%\textbf{SPAdes}&68,189&63,721 \\
%\hline
%\textbf{Megahit}&41,427&39,206 \\
%\hline
%\end{tabular}
%\label{table:n50-ng50} 
%\end{table}

%-----------------------------Misassembles table 

\begin{table}[t]
\caption{Misassembled Contigs using Identity 99\%}
\centering
\begin{tabular}{|c|c|c|c|}
\hline

\textbf{Assembler}& \textbf{No. of Contigs} &\textbf{No.of Bases} \\ [0.5ex] % inserts table %heading
\hline
\textbf{IDBA-UD}& 4,980 (24.91\%) &112,309,828\\
\hline
\textbf{SPAdes} & 3,143 (20.60\%) &108,969,624 \\
\hline
\textbf{Megahit}& 5,977 (21.61\%) &90,889,558  \\
\hline
\end{tabular}
\label{table:misassembled} 
\end{table}

%---------------------------------------------------------------------------------------------------

%%%Remove this line for now \subsection*{Figures}
 
\begin{figure}[!h]
\centering
\includegraphics[width=0.45\textwidth]{CoverageProfile.pdf}  
\caption{\label{fig:coverage-profile} Cumulative coverage profile for the reference metagenome, based on read mapping. }
\end{figure}

%\begin{figure}[!h]
%\centering
%\includegraphics[width=0.45\textwidth]{UnalignmentHistogram.pdf} %to be chamged to qc.coverage-profile
%\caption{Mapping unaligned reads to reference genome using identity 99\%, and Ambiguous Approach CTB This can't be right :)}
%\label{fig:unaligned-reads}
%\end{figure}

\begin{figure}[!ht]
\centering
\includegraphics[width=9.5cm,height=9.5cm]{min-contig-analysis.pdf}  
\caption{\label{fig:min-contig-analysis} Genome fraction, duplication ratio, Contigs overlap ratio, and number of contigs using different minimum contig length,  identity 99\%, and Ambiguous Approach}
\end{figure}

\begin{figure}[!h]
\centering
\includegraphics[width=0.4\textwidth]{CommonUncoveredCoverageProfile.pdf} 
\caption{Cumulative read coverage for bases in the reference metagenome missing from all three assemblies.}
\label{fig:CommonUncovered}
\end{figure}



% \begin{figure}[!h]
% \centering
% \includegraphics[width=0.4\textwidth]{MapQuality.png}  
% \caption{\label{fig:mapquality} Mapping Quality for Quality Filtered Reads to the Reference Genome}
% \end{figure}

% \begin{figure}[!h]
% \centering
% \includegraphics[height=2.6cm,width=0.4\textwidth]{basequality.png}  
% \caption{\label{fig:basequality} Mean Base Quality}
% \end{figure}

% \begin{figure}[!h]
% \centering
% \includegraphics[height=2.6cm, width=0.4\textwidth]{errorprofile30.png}  
% \caption{\label{fig:errorprofile30}Error Profiles for MAPQ >=30}
% \end{figure}

% \begin{figure}[!h]
% \centering
% \includegraphics[height=2.6cm, width=0.4\textwidth]{nucleotidecomposition.png}  
% \caption{\label{fig:nucleotidecomposition.png}Nucleotide Composition for MAPQ >=30}
% \end{figure}
% \subsection*{Mathematics}

 
% \subsection*{Mathematics}

\subsection*{Datasets}
%Numbers in this section are up to date 22 december 2016

We used a diverse mock community data set constructed by pooling DNA
from 64 species of bacteria and archaea and sequencing them with
Illumina HiSeq.  The raw data set consisted of 109,629,496 reads from
Illumina HiSeq 101 bp paired-end sequencing (2x101) with an untrimmed
total length of 11.07 Gbp and an estimated fragment size of 380 bp
\cite{podar}.
 
The original reads are available through the NCBI Sequence Read
Archive at Accession SRX200676.
We received the 64 reference genomes from the original authors. They
consist of 205.6 Mbp of assembled genomes in 64 contigs, and are
available for download at
https://dx.doi.org/10.6084/m9.figshare.1506873.v2

\section*{Methods}
The analysis code and run scripts for this paper are available at:
https://github.com/dib-lab/2015-metagenome-assembly/.

\subsection*{Quality Filtering} 

We removed adapters with Trimmomatic v0.30 in paired-end mode with the
Truseq adapters \cite{trimmomatic}, using light quality score trimming
as recommended in @cite MacManes 2014.

%We next used the fastq\_quality\_filter from the FASTX-Toolkit v0.0.13.2  \cite{FXtoolkit} to remove sequences using the parameters  {\tt{-Q33   -q 30 --p 50}} , which keeps all sequences with  50\%  or more bases with quality score greater than or equal to 30.

\subsection*{Reference Coverage Profile}

To evaluate how much of the reference metagenome was contained in the read
data, we used {\tt bwa aln} to map reads to the reference genome.  We then
calculated how many reference bases were covered by how many mapped
reads (custom script {\tt coverage-profile.py}).

\subsection*{Assemblers}
We assembled the quality-filtered reads using three different assemblers: IDBA-UD
\cite{idba}, SPAdes \cite{spades}, and MEGAHIT \cite{megahit}.  For
IDBA-UD v1.1.1 \cite{idba}, we used {\tt {--pre\_correction}} to
perform pre-correction before assembly and -r for the pe files.

For SPAdes v3.9.0 \cite{spades}, we used { \tt {--meta --pe1-12
    --pe1-s}} where {\tt{--meta}} is recommended when working with
metagenomic data sets, {\tt{--pe1-12}} specifies the interlaced reads
for the first paired-end library, and {\tt{--pe1-s}} provides the
orphan reads remaining from quality trimming.

%@SAM this is what we used?  
For MEGAHIT \cite{megahit}, we used -l 101 {\tt{-m 3e9
    --cpu-only}} where {\tt -l} is for maximum read length, {\tt -m} is
for max memory in bytes to be used in constructing the graph, and {\tt
  {--cpu-only}} to use only the CPU and not the GPU. We also used {\tt
  {--presets meta-large}} for large and complex metagenomes, and {\tt
  {--12} } and {\tt{-r}} are parameters that specify the
interleaved-paired-end and single-end files respectively.

All three assemblies were executed on the same high-memory buy-in node
on the Michigan State University High Performance Compute Cluster, and
we recorded RAM and CPU time of each assembly job using the {\tt
  qstat} utility at the end of each run.

Unless otherwise mentioned, we eliminated all contigs less than 500 bp
from each assembly prior to further analysis.

\subsection*{Assembly analysis using Nucmer}

We used the Nucmer tool from MUMmer3.23 \cite{mummer3.0} to align
assemblies to the reference genome with options {\tt \--coords} {\tt
  -p}. Then we parsed the generated ``.coords'' file using a custom
script {\tt{analyze\_assembly.py}} to calculate several analysis
metrics at two alignment identities, 95\% and 99\%.

\subsection*{Mapping}

We aligned all quality-filtered reads to the reference metagenome with
bwa aln (v0.7.7.r441) \cite{bwa}. We aligned paired-end and orphaned
reads separately using bwa aln samse. We then used samtools (v0.1.19)
\cite{sam-stools} to convert SAM files to BAM files for both
paired-end and orphaned reads. To count the unaligned reads, we
included only those records with the ``4'' flag in the SAM files
\cite{sam-stools}.
 

To extract the reads that contribute to unaligned contigs, we mapped
the quality filtered reads to the unaligned contigs using bwa aln
(v0.7.7.r441) \cite{bwa}.  Then we used samtools to retrieve the reads
that mapped to the unaligned contigs.


%We found chimeric alignments (alignments that split reads in two or more) with the bwa mem aligner using the default parameters (v0.7.7.r441). 

%To count the chimeric alignments, we count the secondary alignments (records with the ``SA" flag) in the SAM file \cite{samtools}. 

%We used SamStats \cite{samstats} to analyze the quality of mapping and errors. 

\subsection*{k-mer Presence}
In order to examine k-mer presence for a k-mer size of 20, we build a
k-mer counting table from the given quality filtered reads using
{\tt{load-into-counting.py} } from khmer (@cite khmer). Then we
calculate abundance distribution of the k-mers in the quality filtered
reads using the pre-made k-mer counting table using
{\tt{abundance-dist.py}}. We followed the same approach to examine
k-mer presence in assemblies.

\subsection*{Gene annotations using Prokka}
We used prokka \cite{prokka} to annotate the reference genome using
{\tt{--outdir mprokka --prefix testasm --metagenome}}. Then we parsed
the testasm.tbl output file to get the coordinates of CDS genes. We then
searched the alignments for how many genes were contained in those
alignments.

\subsection*{Analyzing Assembly: Ambiguous, Best-Hit, and No misassemblies Approaches}
We processed the alignments in three different ways: ambiguous,
best-hit, and no-misassemblies.

% In the ambiguous approach, we keep all alignments, including alignments where multiple contigs align to the same region in the reference, and multiple parts of the same contig align to the same reference region. Alignments to different parts of the reference are kept too. 

% In the best-hit approach, when different parts of one contig align to the reference, we only take into account the highest-scoring alignment and discard the other alignments from that contig.

% In the no-misassemblies approach, we count only contigs that align uniquely and completely to the reference. If the contig aligns partially, or different parts of the contig align to different regions in the reference, we discard all of the alignments.


In the ambiguous approach, we took into account all the alignments of
a contig to the reference, even if alignments overlap in the reference
or they are aligned to multiple locations in the reference.

In the best-hit approach, among all alignments of a contig, we took
into consideration only the alignment with the best score.

In the no-misassemblies approach, we only counted contigs that have
precisely one alignment to the reference.

In all approaches, we flag a base in the reference genome as
``covered'' if it is contained in a kept alignment.  We define the
duplication ratio as the percentages of bases in the reference covered
by two or more kept alignments. We define misassemblies as
those contigs that are divided into different parts when mapped to the
reference.  The number of misassembled contigs is equal to the number
of aligned contigs (both totally and partially) in the ambiguous
approach, minus the number of aligned contigs in the no-misassemblies
approach.

%We define the contig overlap ratio as the number of bases
%aligned to the reference that exist in more than one contig.  We
%define the contig overlap ratio as the number of bases aligned to the
%reference that exist in more than one kept alignment.

All approaches have a non-zero duplication ratio within the reference
because we do not explicitly discard contigs that map to the same
location in the reference.

\section*{Results}

\subsection*{The raw data is high quality}

We trimmed sequences as described in Methods. We retained 7.1 Gbp in
108,422,358 paired-end sequences, and 36 Mbp in 520,403 orphaned
reads.  This quality trimmed ("QC") data set was used as the basis for
all further analyses.

%\subsection*{Mapping Quality and Error Profiles}

%%Might need to be merged with previous section  @CTB
%We used SAMStat \cite{samstat} to analyze the mapping quality and error profiles of quality filtered reads mapped to the reference genome. Figure \ref{fig:mapquality} shows number of alignments in various mapping quality (MAPQ) intervals and number of unmapped sequences. The percentage and number of alignments in each category is given in brackets. 
%Figure \ref{fig:basequality} shows mean base quality of reads with low and high mapping quality. %Figure \ref{fig:errorprofile30} shows the error profile for MAPQ $\geq 30$ and Figure \ref{fig:nucleotidecomposition.png} shows the nucleotide compositions. We also found 309,414 chimeric reads. 

\subsection*{98\% or more of the reference is present in the read data set}

We next evaluated the fraction of the reference genome covered by at least
one read (see Methods for details). Quality filtered reads cover
203,058,414.0 (98.76\%) bases of the reference metagenome (205,603,715
bp total size).  Figure \ref{fig:coverage-profile} shows the
cumulative coverage profile of the reference metagenome, and the
percentage of bases with that coverage. Most of the reference
metagenome was covered at least minimally; only 3.33\% of the
reference metagenome had mapping coverage \textless 5, and 1.24\% of
the bases in the reference were not covered by any reads in the QC data
set.

In order to evaluate reconstructability with De Bruijn graph
assemblers, we next examined k-mer presence for a k-mer size of 20. Of
the 174m 20-mers in the reference data set, 98.7\% were present in the
data set and 95.5\% of them occurred with abundance 5 or greater in
the quality filtered read data set.

%Data source note: the 95.5 is from
% SRR606249.qc.dist (1-0.045)*100 , the 98.7 is from the same file but
% (1-0.013) *100

\subsection*{MEGAHIT is the fastest and lowest-memory assembler evaluated}
%Numbers in this section are up to date 22 december 2016

We ran three commonly used metagenome assemblers on the QC data set:
IDBA-UD, SPAdes, and Megahit. We recorded the time and memory usage of
each (Table \ref{table:time-memory}).  MEGAHIT outperformed both
SPAdes and IDBA-UD considerably, producing an assembly in one hour --
approximately 17 times faster than IDBA and 42 times faster than
SPAdes.  MEGAHIT used only 34.4 GB of RAM -- 1/5 to 1/11th
the memory used by IDBA and SPAdes, respectively.

\subsection*{Much of the reference is covered by the assemblies}
%Numbers in this section are up to date 22 december 2016

We next evaluated the extent to which the assembled contigs recovered the
``known/true'' metagenome sequence by aligning each assembly to the
reference (Table ~\ref{table:coverage-analysis}).  All three
assemblers generate contigs that cover more than 89\% of the reference
metagenome at high identity (99\%) with little duplication
(0.55 -1.0\%) (see "Ambiguous approach" in
Table~\ref{table:coverage-analysis}).  If we relax the identity
threshold to 95\%, then the assembled contigs cover more than 94\% of
the reference metagenome.

% @CTB note for discussion: so here the assemblies approximate the
% best possible, as measured by high-coverage bases/k-mers.

When we use only the highest-scoring alignment at 99\% identity, we
find that the reference coverage drops to 57-68\%, depending on the
assembler ("Best hit approach",
Table~\ref{table:coverage-analysis}). The reference coverage from
highest-scoring alignments does not substantially increase when the
identity threshold is relaxed to 95\%.

At 99\% identity with the ambiguous approach, approximately 6.23\% of
the reference is covered by no contig from any of the three
assemblies; we discuss this in more detail below.
%common uncovered / no.of bases in the reference *100
% @CTB we need to extend methods to contain mention of common uncovered.
% Probably belongs in the ``Analyzing Assembly'' section.

\subsection*{The generated contigs are broadly accurate} 
%Numbers in this section are up to date 22 december 2016

When counting only the best alignment per contig at a 99\%
identity threshold, more than 72\% of contigs align to the reference
completely, i.e. across the whole length of the contig (Table~\ref{table:contigs-analysis}, ``Best hit'', Totally Aligned Contigs \%).  If we allow
multiple alignments per contig, then more than 80\% of contigs align
completely (although not contiguously) to the reference (Table~\ref{table:contigs-analysis}, ``Ambiguous'', Totally Aligned Contigs \%).
Approximately 4-9\% of the contigs align only partially, and the
remaining 8 - 10\% of contigs do not align at all to the reference
(discussed in detail below).

\subsection*{The assemblies contain most of the raw data}
%Numbers in this section are up to date 22 december 2016

The assemblies also represent the majority of the reads (99.7\%) and
the majority of the abundance-5 20-mers (93.7\% of the 198m
high-abundance 20-mers in the reads), suggesting that the assemblies
represent the underlying content of the reads very well.
% %the 93.7 is from qc500.dist (1-0.027)*100.
% the 99.7% is from  (1-0.003)*100

\subsection*{Most genes within the reference metagenome are contained within contigs}
%Numbers in this section are up to date 21 december 2016

The reference genome has 188,880 CDS with 91,806 annotated as genes,
based on a Prokka annotation (see Methods).
Using the ``ambiguous'' alignment approach and 99\% identity, the
IDBA-UD assembly contained 82,791 (95.20\%) of the reference genes,
while SPAdes contained 83,475 (94.44\%), and
MEGAHIT contained 80,256 (94.59\%) of the reference gene
coordinates.

%These numbers are from prokka-analysis.out.
%also see prokka.out and mprokka/testasm.tbl
% @CTB are these in a table anywhere?

% second tier todo - @CTB what about taking the nucleotide sequences
% output by prokka and see if they match in the assembly?

\subsection*{The portions of the reference metagenome that are not reconstructed are not present in the read data set}
%Numbers in this section are up to date 27 December 2016
 
We identified XX bases in the reference that had no match (at 99\%
identity with the ambiguous mapping approach) in any of the
assemblies, and evaluated their base coverage. XX (19.7\%) had no coverage
in the reads, and a total of XX (29.2\%) had coverage less than 5.

% @CTB update with number of bases; check numbers in source files.
% @CTB what can we say about the rest? anything?
%(Using awk '$1 <5 {sum +=
%$2} END {print sum}' common-uncovered-coverage.out)/no. common
%  uncovered bases from analyze_assembly.py output *100

\subsection*{Large portions of several reference genomes are not assembled by any assembler}
%Numbers in this section are up to date 21 december 2016

A number of the genomes in the reference metagenome had many missing
bases in the assemblies (Table ~\ref{table:genomes_uncovered}). In
three extreme cases, Shewanella baltica\_OS223, Fusobacterium
nucleatum, and Desulvovibrio contribute, 18.25\%, 16.95\%, and 12.85\%
of the common uncovered bases respectively.

%these numbers are after sorting file QC.AMBIGUOUS.99.uncovered using
%[sort -n -k2 -r  QC.AMBIGUOUS.99.uncovered>QC.AMBIGUOUS.99.uncovered.sorted ] and the percentage is computed w.r.t no. of common uncovered bases

% @CTB: we need to add in percentages of the genome, I think, as well.

\subsection*{Many assembled contigs do not align to the reference metagenome}
Depending on assembler, between 8.04\% and 10.87\% of the assembled
contigs do not align anywhere in the reference metagenome.

6.49\% of reads mapped to the unaligned contigs of IDBA. Only 33.01\%
of those reads mapped to the reference. (2.14\% of all the reads).
6.23\% of reads mapped to the unaligned contigs of SPAdes. Only
28.97\% of those reads mapped to the reference. (1.80\% of all the
reads).  5.87\% of reads mapped to the unaligned contigs of
Megahit. Only 24.80\% of those reads mapped to the reference. (1.45\%
of all the reads)
%The above numbers are coming from
%iqc.AM99.aligned.reads and iqc.AM99-mapped-reads for IDBA,
%sqc.AM99.aligned.reads and sqc.AM99-mapped-reads for SPADES, and
%mqc.AM99.aligned.reads and mqc.AM99-mapped-reads for megahit.
For
each assembly, approximately 5m quality-filtered reads map only to the
unaligned contigs (and nowhere in the reference).
%Numbers from subtracting numbers in those files:
%(iqc.AM99.aligned.reads - iqc.AM99-mapped-reads) for IDBA
%(sqc.AM99.aligned.reads - sqc.AM99-mapped-reads) for SPAdes 
%(mqc.AM99.aligned.reads - mqc.AM99-mapped-reads) for megahit 

 

For IDBA QC, the reads that aligned to the unaligned contigs but not
to the reference has coverage bases 27.21\% \textless 5 in the
unaligned contigs. 75.72\% has coverage \textgreater 0.

For SPAdes QC, the reads that aligned to the unaligned contigs but not
to the reference has coverage bases (3,109,539) 24.04\% \textless 5 in
the unaligned contigs. 79.10\% has coverage \textgreater 0.

For Megahit QC, the reads that aligned to the unaligned contigs but
not to the reference has coverage bases (2,578,896) 20.18\% \textless
5 in the unaligned contigs. 83.08\% has coverage \textgreater 0.

%The above numbers are comming from iqc.unmapped.out, sqc.unmapped.out, and mqc.unmapped.out
 
%(What are these? CTB to evaluate with MetaPalette.) @CTB I couldnot do it yet

\subsection*{All three assemblers recover most of the reference}
%Updated on 27 December  

%we have 6.23% of the reference is commonly uncovered. We have 0.64% of bases covered by IDBA only, 0.74% of bases covered by SPAdes only, and 1.92% of bases covered by Megahit only.  Each uniquely fail to recover about 1\% (1.22-1.79\%) of the reference. XX is roughly.

XX\% of the reference metagenome is recovered by all three assemblers
(“common covered”) with relatively little duplication. (99\% identity,
ambiguous alignments allowed).
 
\subsection*{All three assemblers fail to recover 6.23\%  of the reference}
  

(2,518,234) of the common uncovered bases has zero coverage (19.66\%),
while (3,739,532) of the common uncovered bases has coverage \textless
5 (29.2\%).

\subsection*{At the margins, the three assemblers differ at about 1\% in recovery}
IDBA, SPAdes, and MEGAHIT each uniquely recover about 1\% (0.64
-1.92\%) of the reference metagenome, and each uniquely fail to
recover about 1\% (1.22-1.79\%) of the reference.

\subsection*{Pending text: To add or not}
Figure \ref{fig:min-contig-analysis} shows the number of contigs using
different minimum contigs cutoff.  The figure shows that Megahit
\cite{megahit} has more fragmented contigs in the assembly. The small
contigs size leads to best alignment; Megahit has the highest number
of uniquely covered bases, highest genome coverage, and lowest
unalignment.

\section*{Discussion}
\subsection*{Assembly recovers basic content well}
%Updated on 26-28 December 
The majority of the reference metagenome is recovered by all three
assemblers: 89\% or more of the reference metagenome is contained
within each assembler’s output at 99\% identity, and 94\% can be
recovered if we relax the identity threshold to 95\%.  This is close
to the measured maximum reconstructability based on read mapping and
k-mer presence: 98.76\% of the reference is covered by at least one
read, and 98.7\% of the genome is present in k-mers of size 20. %98.76
and 98.7 are from SRR606249.qc.coverage and SRR606249.qc.dist
respectively

The contigs generated align well to the reference metagenome: 80\% (or
more) of contigs align to the reference metagenome across the whole
length of the contig, and another 4\% (or more) of the contigs are
entirely contained within the reference metagenome, although they do
not align to only one location in the reference. This could be caused
by misassembly (a computational error) or by rearrangements in the
source DNA (in which case the reference is incorrect).
 
%(Talk about gene content of assemblies here.)
%Updated on 27 December 
More than 6.23\% of the reference metagenome is missing from all of
the assemblies when 99\% similarity is required, and large portions of
the missing sequence are from a few genomes -- in some cases, close to
a third of the source genome is missing from the assemblies.
%6.23 is the (12,807,937 bases)/205603715 *100: all coming from assemblies.stats.QC.AMBIGUOUS.99 

 %932,873 out of these missing bases are GC bases.  The read coverage
of the unassembled regions of the source genome is low to nonexistent
, while other portions of the source genomes are well represented
within the reads; these regions were simply not sequenced, either
because they were missing from the input DNA or because they challenge
the sequencer. %However, some unassembled regions have good coverage:
we have only 29.2% with coverage <5 in the common uncovered.


\subsection*{Assembly produces content not in the reference metagenome}
%@CTB: I don't know what XX in this section refering to. 
% XX was >5 is comming from iqc.unmapped.out, sqc.unmapped.out, and mqc.unmapped.out where ~75% of the reads mapped to the aligned contigs and not to the reference has coverage of one at least in the contigs, but the text refers to ccontigs coverage? so something here doesn't add up;

In addition to recovering most of the content of the reference
metagenome, all three assemblers generate many contigs (approximately
10\% of the total) that do not map to anything in the reference.
While some could be misassemblies, many of the reads that map to these
contigs do not map anywhere in the reference, while many of the
contigs have coverage >XX, suggesting that they are present within the
source DNA at an abundance that is similar to many of the genomes in
the mock community.

%This part I left as it is.

Whatever their true identity, these contigs are not present in the
reference metagenome. Because this is a mock metagenome for which
isolates were grown and individually extracted prior to being combined
for sequencing, we believe that these extra genomic contigs must come
from one or more contaminants.  We cannot rule out contamination from
kits, but because large amounts of DNA were used to create the mock
community, it seems unlikely that these are due to trace contaminants
created by PCR (as can happen in amplicon studies); they are most
likely from contaminants grown together with the isolates.

\subsection*{Different assemblers differ, but not by much}
%Updated on 24 December :@CTB still need the XX from: we have 6.23% of the reference is commonly uncovered. We have 0.64% of bases covered by IDBA only, 0.74% of bases covered by SPAdes only, and 1.92% of bases covered by Megahit only.  Each uniquely fail to recover about 1\% (1.22-1.79\%) of the reference.

The three assemblers differ very little in recovery of basic content,
as judged by reference metagenome alignments: at 99\% identity, XX\%
is recovered in common, with 0.64 to 1.92\% specific to each assembler
(and a total of XX\% being recovered by one or two, but not all three,
of the assemblers). None of the assemblers include all of the sequence
produced by the others. %, and the extra sensitivity offered by
%MEGAHIT comes with a higher level of duplication.
%I commented the previous sentence as Megahit is not the lowest in duplication 

In practice, then, we do not see a significant advantage to one
assembler over another in terms of recovering the known reference, and
we speculate that (for this data set), there is little advantage to
combining assemblies from multiple assemblers, since very little extra
content will be recovered.


\section*{Conclusions}

Assembly works well. There is no big difference between assemblers'
performance in terms of assembly quality. In terms of cost, Megahit is
much faster and utilizes less memory.

% @CTB mention long reads, microdiversity (banfield study).
% consequences for genome binning etc.

% do we make comments about tradeoff between single contig analysis,
% quast analysis, etc?
% better misassembly analysis?

\subsection*{Author contributions}
In order to give appropriate credit to each author of an article, the
individual contributions of each author to the manuscript should be
detailed in this section. We recommend using author initials and then
stating briefly how they contributed.

\subsection*{Competing interests}
No competing interest to our knowledge.

\subsection*{Grant information}
This work is funded by Moore and NIH.

\subsection*{Acknowledgments}
Michael R. Crusoe

{\small\bibliographystyle{unsrtnat} \bibliography{references}}

\bigskip
% References can be listed in any standard referencing style that uses a numbering system
% (i.e. not Harvard referencing style), and should be consistent between references within
% a given article.

% Reference management systems such as Zotero provide options for exporting bibliographies as Bib\TeX{} files. Bib\TeX{} is a bibliographic tool that is used with \LaTeX{} to help organize the user's references and create a bibliography. This template contains an example of such a file, \texttt{references.bib}, which can be replaced with your own. Use the \verb|\cite| command  to create in-text citations, like this .


% See this guide for more information on BibTeX:
% http://libguides.mit.edu/content.php?pid=55482&sid=406343

% For more author guidance please see:
% http://f1000research.com/author-guidelines


% When all authors are happy with the paper, use the 
% ‘Submit to F1000Research' button from the menu above
% to submit directly to the open life science journal F1000Research.

% Please note that this template results in a draft pre-submission PDF document.
% Articles will be professionally typeset when accepted for publication.

% We hope you find the F1000Research Overleaf template useful,
% please let us know if you have any feedback using the help menu above.


\end{document}

